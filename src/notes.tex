% this file contains some notes

Artefact: a ABM agent based model, models the last mile in the food delivery business with customers, restaurants and deliverers.
The model can be configured to simulate the delivery process, in terms of the nr of agents of each type, the number of meals delivered
over a period of time.
It models the actual behavior of individual agents, to measure their combined effect.

Context: uses an existing computer modelling environment: NetLogo, the model will run on a pc, the program can be adjusted by setting some parameters.
It can also be run by using a python program.
It is used to analyse a real world problem, but in a higly abstracted way, using only some key elements from th reality.

Impact: not applicable,

Stakeholders: potential users,

Design problems:
- take into account other users of this model, program it in a neat fashion and document
- must reflect real delivery and agents decisions
- it could be programmed in several maners
- delivers a sufficient artefact.
- endusers may want to use the model as is, or may want to change the code.

Social context:
-the developer creating the system
-other users, students

Knowledge problems:
- emperical:  run simulations, gather data
- analytic: think about the gathered data.
- other articles, data, economic theories

Design goal: the abm
Knowledge goal: what are consequences of changing some strategies of agents, understanding a real world problem


Design questions
-Improve- the knowledge about: effects of deliverer employment strategies in the food deliveries last mile
-by- simulating the strategies in this setting with an ABM in NetLogo with only some key properties
-such that-
the total number of prepared meals can be calculated after some time given a strategy,
-in order to- give an explanation why companies favor a certain strategy.

Knowledge questions:
Empirical questions: observations
Analytical questions: reasoning

What are the key properties of the agents?
What are the effects of these properties?

The context: the food deliveries last mile, with customers, restaurants and deliverers and 1 company.

In de tekst:
vergelijk de agents met die uit de echte wereld, met verschillen/overeenkomsten

engineering cycle:
->  -Design cycle
|     ->    1. problem investigation: -Improve the knowledge about: effects of deliverer employment strategies in the food deliveries last mile
|     |     2: treatment design: the artifact, the abm model
|     |<-   3: treatment validation: testing the model, run it with some settings, are the results expected?
|   -Implementation
|<- -Evaluation

The context of the artefact: a researcher running the simulation on its pc, setting parameters, changing code, running it with python

Requirements, should simulate key elements of the real world.
Restricted to the key elements and eliminate other factors
\todo: add documents to the python projects, with requirements.

The stake holders: decision makers at the food delivery companies, other researchers interested in using ABM.

Describe the validation of the artifact: average orders per week, do get all deliverers get a turn, does no driver get stuck.

