
1 Formulate the research question,
2 formulate the research design,
3 find an instrument to collect data,
4 select your sample,
5 write a research proposal,


Food delivery companies, like Uber Eats and Thuisbezorgd, offer a service where customers can order food from restaurants and get the order delivered at home.
These companies get a fee per delivery and pay deliverers an hourly wage and/or pay them per delivery (commission based).

Lately delivery companies have had some bad publicity of being unfair and exploiting the deliverers.
That is why in the Netherlands some companies hire them as employees with all benefits and pay an hourly wage.
Other companies still see them as independent workers and get paid on commission.

To make a profit, delivery companies need as many of deliveries as possible, for this they need customers ordering food and for that restaurants offering food.
The delivery company also has to make sure the ordered meals get delivered on time, customers don't want cooled down food or have to wait too long,
for this they need enough deliverers.
Not enough deliverers means not all orders are delivered or delivered on time, this makes restaurants and customers unhappy and may stop using their service.
Too many deliverers, they have not enough to do and cost too much for the company when payed an hourly wage.
Deliverers base their happiness on the distances they travel and on the wage they earn, if unhappy they may stop being a deliverer.

To orchestrate the process of ordering food by customers, handle the orders by restaurants and deliver the prepared meals by deliverers an system is needed.

This system can offer different services:
- order food from a restaurant
- a way to devide ordered meals among deliverers
- a way to help deliverers find a route
- advertise a deliverer is neede for delivery
- keep track of the deliverers on route
- calculate commision, and a panalty for being late
- register users: customer, restaurant or deliverer.
- cancel a user

Before building this system, insight is needed on how to support users.
One could just offer a simple service and let the market decide.
Anyone could become a user: restaurants, consumers and let deliverers decide for themselves to pickup a meal or not.
Or regulate more, restrict the number of deliverers and destribute meals among deliverers.
When distributing meals some selection method is needed, e.g., the nearest free deliverer to the restaurant.

To gain this insight I propose to build a model using NetLogo to simulate the process of restaurants ready to take orders, customers ordering meals,
restaurants preparing the meals, assigning the ready meals to the deliverers, the traveling of the deliverers and finally the deliverie.
The simulation will look at profit for the company, users entering and leaving (when unsatisfied) over en period of times.


create models

describe the model and describe the build process

use the model to investigate 2 scenarios.



Paper has 2 parts

Research consists of:

1: The model, simulation a real world event. design cycle

2: The model is tested and used, presents the results. empirical cycle

Applied research

no purposefully bias
no chery picking
no subjectivity
must be scientific
1 The research was conducted in a controlled environment
2 Validate employed methods, justify application
3 It is systematic
4 Must be repeatable
5 The conclusions must be based on observations and information
6 The research method must be able to survive critical review

Classification

Descriptive research aims to get an overview of the current situation.

Correlational research aims to find associations between observations: if
the value of one variable increases, does the value of another variable
increase at the same time?

In explanatory research the aim is not to only find an association, but also
to find an explanation for the observed association. Finding such an
explanation is not always trivial, as we will demonstrate in a moment.

Exploratory research aims to assess a situation. Often it involves
research areas lacking extensive existing knowledge

Studies can combine exploratory, explanatory, correlational and
descriptive research, and this is even recommended.

Theorie before: profit maximisation, how wil agents react, must be determined before,
theory of agents behaoviour
result adjusted theory


theories serve to describe, explain and predict how
the artefact will interact with the intended context.

The context for using the artefact: a researcher who may use the netlogo model and adjust it to its needs.
The researcher wil download the model, open it in NetLogo, change some settings, run it on a computer, adjust it to by changing the code.
It is a model based on some common sense assumptions and multiple agents behavior described in the literature, that may be adjusted or made more elaborate.

The inpact of this research is low, it could though give some ideas about how to pay deliverers (a wage or per delivery)

The model could be made more advanced, e.g. including sickleafs, holidays, maybe include some game theory.

Stakeholders: other ai researchers or economists, document teh model

May solve problems in the real world

Design problems => Design questions -> design goals => create the artefact
Knowledge problems => knowledge questions (questions about the artefact, the world, and the artefact within the world) -> knowledge goals  to answer the goals build the artefact
- empirical knowledge questions, which require data to be solved
- analytical knowledge questions, which require reasoning to be solved

the design questions are influenced by knowledge questions

questions are more specific then problems
Normally, problems are defined first and questions

knowledge questions: empirical from data, after running the simulations
analytical questions:  think of how the agents could act, what are their behavior rules. descriptive as well as explanatory












\section{Background}
The pickup and delivery problem is a well known real world problem.
In this age, online shopping has become a mayor way of customer expenditure.
But all these purchases have to be delivered to the customers doorsteps.
This causes a lot of traffic and movements of vans and bicycles through the streets.

For the delivery people it is key to make the deliveries as efficient as possible.
This saves time and energy, which is beneficial for their profit and for the environment.

Efficiency means to find a short route between a pickup point and delivery point.
It may also mean to have a delivery route in which several parcels can be delivered in a row and finding the shortest total distance.
A common constraint is to pick up a parcel on time and deliver it before a due time.
But things can go wrong on the way, street works, road congestions these may slow down deliveries.

The pickup and delivery problem has been studied alot.

My proposal is to study some of the behavior of delivery people in the situation where they can enter or leave the market.
This is often the case in reality, drivers are mostly independent subcontractors when working for companies like Uber or PostNl
Once they are drivers they want to make a profit and find efficient routes between pickup points and deliveries.

If there are to many drivers the profit will be lower per driver, but if there are not enough maybe not all parcels can be delivered

If drivers can not make a profit they quit.

The goal is finding an optimal amount of drivers for the market and a way for drivers to find an optimal route.

For this an experiment will be build in NetLogo.
This will consist of a grid of roads, pickup points, random delivery points and agents representing drivers.
Together with a set of rules series of runs will be executed with different settings.
Rules can be:
-only one driver at a time at pickup and delivery points,
-a driver can

What will happen if there are no regulations and the market has full control, and what if a maximum amount of drivers is allowed.
- every agent maximises its own profit
- a supervisor maximises total profit

Model:
delivery guys
restaurants
customers.

There is one delivery company who has an app that deliverers, customers and restaurants can use.
Any person can become a deliverer by installing an app and make an account.
This has an analogy with companies like Uber.
If they don't make any money they will leave.
Any restaurant can use the app, but if they are not satisfied they will stop using it.
Any person can order food by using the app, but if not satisfied they will leave.
The deliverer company makes money by charging a fee per delivery for the restaurant and the customer.
It pays a part of the fee to the deliverer, but only when delivered on time.
But if an order is delivered to late they don't get paid, and the restaurant may stop using the app.
If a delivery is not delivered to a customer they will stop using the app, also the restaurant will stop using the app.

The app company may hava a strategy to leave control to the market, only offer the app.
Orders are assigned to the first deliverer who shows up.
Or they may intervene, limit the number of deliverers, and devide orders among the delivers via a system.







If to many drivers they will leave, if not enough restaurants an customers will become unsatisfied.


% curiosity driven research

Develop app for food delivery what features should be include to maximise profit.
Thinks like allow a total free market or impose rules to regulate.



\section{Research questions}

\begin{description}
    \item[RQ1.] What are key variables (e.g., the number of deliverers or the order distribution method), in maximising the profit for a food delivery app?
    \item[RQ2.] What are the relative effects of changing these variables? Can some optimal settings be found?
\end{description}

\section{Research methods}
The research method consists of buiding a multi-agent model in the NetLogo application.
First the rules in this multi agent model have to be found and decided.
The model needs rules for deliverers to deside when to accept a pickup, based on the distance to the restaurant and from the restaurant to the delivery point.
If the distance is too long they may deside not to pick it up.
Other rules are who gets the order, it could be that who arrives first gets the order.

\subsection{Design}
-address a problem in the world: how to manage the process of deliveries with 3 types of users (deliverers, restaurants, customers).
-endusers delivery companies, other researchers
-
This including building the model in NetLogo.
It involves learning the programming languages, search for existing models that can be used.
It also involves finding behavior rules for the agents and variables that can be set.

Design questions: how to program behavior of the agents, how to model an abstraction of the world (roads, restaurants, customers, deliverers),
which variables can a user of the model set,

Knowledge questions: what methods do existing delivery companies use.
What methods exist to determine the shortest distance between 2 points.
What rules may agents, in the context of food delivery, have to base their behavior on.
What are the effects of setting variables by the user.

design goal: create a model that simulates agents behavior
knowledge goal: gain knowledge about agents behavior

instrument design, netlogo model is an instrument
\subsection{Procedure}

\subsection{Data analysis}
After the model is build, several tests have to be done.
A test

\section{Time schedule}


\section{Risk analysis}


\bibliography{bibliography}


\end{document}




Firstly, economic models have been conceived of as idealized entities.
From this perspective economists are seen to make use of stylized, simplifying,
and even distorting assumptions as regards the real economies in their modelling
activities

Secondly, it has been suggested that models in economics are various
kinds of purpose built constructions: some are considered to have representational
status, others are considered as purely fictional or artificial entities.



Idealization is typically potrayed as a process that starts
with the complicated world with the aim of simplifying it and isolating a small part
of it for model representation;

The basic idea that philosophers of economics have derived from Mill is to conceive
of models as abstracting causally relevant capacities or factors of the real world for
the purpose of working out deductively what effects those few isolated capacities
