\section{Related work}

\subsection{Economic modeling: a short introduction}\label{subsec:economic-modeling}
The field of the ABM model created in this study is economics.

Until the late 1970s, modeling in economics consisted mostly of seperated macro-, microeconomic models.
The authors of ~\cite{morgan2012models} categorize them as follows:

\begin{enumerate}
  \item Core micro-economic theory uses sophisticated mathematical methods in modeling economic phenomena.
  \item Macroeconomics relies more on purpose-built models, often devised for policy advice.
\end{enumerate}

In the 1976, Robert E. Lucas Jr.~\cite{lucas1976econometric} wrote an article stating that while some macroeconomic models are good for forcasting
they are not always suitable for evaluating policies.
Changes in policies may lead to a false evaluation because individual agents may adjust their behavior not anticipated in the model~\cite{HURTADO2014S12}.
This is knows as: Lucas critique.

This lead to the development of: Dynamic stochastic general equilibrium (DSGE) models, these models incorporated microeconomics into macroeconomics~\cite{moos2019facts}.
One key element of these models is the existence of stochastic parametric drift, i.e.\ stochastic drift is the change of the average value of a stochastic (random) process \footnote{\url{https://en.wikipedia.org/wiki/Stochastic_drift}}.
The randomness refers to the unexpected behavior of agents over time.

Other economics, think that the DSGE model is too restrictive.
The DSGE model assumes that an economy can reach and sustain an equilibrium.
Newer views state: economies are non-linear, complex dynamic systems, which rarely, if ever, reaches equilibria~\cite{hamill2016agent}
Agent based simulations make it possible to investigate these systems.

\subsection{Agent base modeling in economics}\label{subsec:agent-base-modeling}
The term Agent-Based Modeling (ABM) refers to a class of modeling methods designed for the study of systems whose dynamics are driven by successive interactions among heterogeneous entities~\cite{tesfatsion2023agent}.

Agents have the following properties~\cite{hamill2016agent}:
\begin{enumerate}
  \item Perception: agents can see other agents in their neighbourhood and their environment.
  \item Performance: agents can act, such as moving and communicating.
  \item Memory: agents can recall their past states and actions.
  \item Policy: agents can have rules that determine what they do next.
\end{enumerate}


Models of this kind simulate how artificial agents behave in an artificial environment over a time interval.
These agents can be anything e.g: ants , radioactive nuclei, viruses, as long as they have some properties that can change over time, and they respond to other agents and their environment.

The field of agents in this study is more related to economics, planning and game-theory.



mschrijving wat modellen zijn en geschiedenis


