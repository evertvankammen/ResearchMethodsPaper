\section{Results}


Consumers order meals from restaurants they like, if a meal is delivered cold they dislike the restaurant.
If they dont like the restaurant they will not place any orders anymore at that restaurant.

Restaurants create meals, if they dont get any orders for some time they quit.

The delivery provider make money for each order placed via their system, at the end they must have enough deliverers so that
customers keep ordering.
They have to pay the deliverers for the deliveries.



\subsection{Model variant 1}
Here come the results belonging to variant where deliverers are hired and paid an hourly wage.
The company destributes deliveries equally among the hired employees.
The company has to deside on how many to hire and for which periods.


\subsection{Model variant 2}
Here come the results where deliverers are independent contractors.
Deliverers come and go whenever they are pleased, like a open market.
Now deliverers have to deside to become a deliverer, deside to stop and to decide to go for a delivery.
To keep this simple, the meals are destributed among the available deliverers.
If a delivery person does not make enough many during a day they will quit.
