
\section{Introduction}\label{sec:introduction}

Food delivery companies, like Uber Eats and Just Eat Takeaway, offer a service where customers can order food from restaurants and get the meal delivered at home.
These companies get a fee per delivery and pay deliverers an hourly wage and/or pay them per delivery.

E.g., Uber Eats~\footnote{\url{merchants.ubereats.com/gb/en/pricing/}} charges 30\% of the total value of an ordered meal to be paid by the restaurant.
Deliverers get paid per delivery (this is not true in all countries) this consists of a base pay and an optional tip from the customer.
Drivers may decide to take an order or not, they are seen as independent contractors.

Lately delivery companies have had some bad publicity of being unfair and exploiting the deliverers.
In the Netherlands the government calls the status of these deliverers: bogus self-employment.
That is why in the Netherlands delivery companies are forced to hire them as employees with all benefits and pay an hourly wage.
Deliveroo, a British company, even left the Netherlands market for this reason.
But in other countries they still see them as independent workers and pay them per delivery.

Delivery companies seem to prefer the per-delivery system as they do not have to pay for the time no deliveries are made.
This system may have advantages for the deliverers too, some may make more money if they can make many deliveries.
But, if deliverers don't make enough money they may quit and thus fewer deliveries can be made leading to lesser profits for the delivery companies.

In this research project different scenarios are simulated where deliverers deliver ordered meals to customers.
The scenarios differ in how deliverers are treated: as independent contractors who get paid per delivery versus
deliverers hired as employees.
Also, some other variables are examined: how meals are distributed among deliverers, the mean delivery preparation times, how deliveries are distributed among deliverers.
In the context of these scenarios the total number of deliveries of the company will be calculated, also the number of participating restaurants, deliverers and customers will analyse.

This is an economic problem that has both micro and macroeconomic facets.
Microeconomic as it deals with agents making decisions, e.g., hire deliverers, maximise the delivery companies deliveries, order food, become a deliverer or deside to stop being a deliverer.

But macroeconomic elements are also present, but not in the conventional sense.
Macroeconomics is the study of aggregate and emerging statistical regularities in the economy~\cite{cincotti2022we}.
The Agent Based Model, shows the results of the behavior of many individuals as a whole, which are the result of the behavior and interaction of heterogeneous agents.
This differs from conventional macroeconomics that sees agents: with behavior of a `typical' that is `representative' or `average' individual.

A model will, like in a laboratory setting, include only properties we are interested in and exclude elements like e.g., weather and physical conditions of deliverers.

The tool for building the model is: NetLogo ~\cite{NetLogo2024}.
This is a multi-agent programmable modeling environment.
This environment has been used to create models for many sciences concerning multi-agents (from how viruses spread to how agents act in game theory~\cite{r2019agent}).

The environment where multiple agents act is called a multi-agent system end the problem is called a multi-agent planning problem (from~\cite{russell2016artificial}).

\subsection{Research goals}\label{subsec:research-goals}
This research is not of a thoroughly theoretical nature, its more of an exploratory/explanatory nature of ABM in the setting of the last mile of food delivery industry.

That being said: the main research goal is to create an Agent Based Model in NetLogo to compare two deliverer employment scenarios in the food delivery industry.

The two main scenarios of interest are:
\begin{itemize}
    \item a scenario where the deliverers are hired as employees
    \item a scenario where deliverers are independent contractors and get paid per delivery
\end{itemize}

% change this afterwards:
% During the model building and design, new-found insights will be used in deciding which variables to use, and adjust the model as needed.



Thus, the research goals are:

\textbf{Building an Agent Bases Model in NetLogo for the fooddelivery proces, as a highly abstracted but still believable world.}

\textbf{Finding how the number of deliveries for a delivery company varies when using independent contractors versus hiring deliverers as employees.
 And what are the relative importance's of the models variables.}


The model will be build incremental, more about this in the methods section.

The second step is executed from the situation where deliverers are hired and thus their numbers are stable.
Then find the effects of several variables and find settings where the model is stable.

From the stable settings, determine the average deliveries per deliverer per timeunit.

Given these stable settings run the model a second time but now the deliverers may leave when their individual deliveries are less
compared to the first run.

The differences are the answers to the second research goal.



