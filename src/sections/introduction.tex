
\section{Introduction}

Food delivery companies, like Uber Eats and Thuisbezorgd.nl, offer a service where customers can order food from restaurants and get the meal delivered at home.
These companies get a fee per delivery and pay deliverers an hourly wage and/or pay them per delivery.

E.g., Uber Eats~\footnote{\url{https://merchants.ubereats.com/gb/en/pricing/}} charges 30\% of the total value of an ordered meal to be paid by the restaurant.
Deliverers get paid per delivery (this is not true in all countries) this consists of a base pay and an optional tip from the customer.
Drivers may decide to take an order or not, they are seen as independent contractors.

Lately delivery companies have had some bad publicity of being unfair and exploiting the deliverers.
In the Netherlands the government calls the status of these deliverers: bogus self-employment.
That is why in the Netherlands delivery companies are forced to hire them as employees with all benefits and pay an hourly wage.
But in other countries they still see them as independent workers and pay them per delivery.

Delivery companies seem to prefer the per-delivery system as they do not have to pay for the time no deliveries are made.
This system may have advantages for the deliverers too, some may make more money if they can make many deliveries.
But, if deliverers don't make enough money they may quit and thus fewer deliveries can be made leading to lesser profits for the delivery companies.

In this research project different scenarios are simulated where deliverers deliver ordered meals to customers.
The scenarios differ in how deliverers are treated: as independent contractors who get paid per delivery and thus may decide to take an order or not versus
deliverers hired as employees where orders are distributed among available deliverers.
In the context of these scenarios the profit of the company will be calculated, also the number of participating restaurants, deliverers and customers will analyse.

This is an economic problem that has both micro and macroeconomic facets.
Microeconomic as it deals with agents that make decisions, e.g., hire deliverers, maximise the delivery companies profit, order food, become a deliverer or deside to stop being a deliverer.
Macroeconomics, because there is a labour market, i.e., the deliverers.

A model will, like in a laboratory setting, include only properties we are interested in and exclude elements like e.g., weather and physical conditions of deliverers.
The tool for building the model is: NetLogo ~\cite{NetLogo2024}.
This is a multi-agent programmable modeling environment.
This environment has been used to create models for many sciences concerning multi-agents (from how viruses spread to how agents act in game theory~\cite{r2019agent}).

Other research projects reported about the food delivery sector using the NetLogo tool.
An existing model for the food delivery context is created by~\cite{ismail2024software}.
This model was used to research the efficiency of food delivery.
The model consists riders, vendors and customers.
The world consists of a 2-dimensional grid where riders have to pick up food from vendors and deliver it to customers.
This project also mentioned some real world rules the 3 types of agents possess.

In another study ~\cite{antelmi2024reliable} NetLogo is compared to other ABM systems and integrations with other programming languages are analysed.

\subsection{Research goals}
The main research goal is to create an Agent Based Model to compare two deliverer employment scenarios in the food delivery industry.

The two main scenarios of interest are:
\begin{itemize}
    \item one, where the deliverers are hired as employees
    \item versus a scenario where deliverers are independent contractors and get paid per delivery
\end{itemize}

During the model building and design, new-found insights will be used in deciding which variables and, if needed, which economic theories to use, and adjust the model as needed.

This first part of the model will be used to find a baseline and will be used to test the model.
This model is the simpler one, at start the certain variables to be used are the number of deliverers and the strategy for delivery distribution.

The second part is based on the first, the main-difference is that the delivers are not hired but paid per delivery, if they make enough money they stay otherwise they stop.
Interesting variables are: the amount paid per delivery, the wage they could earn in a steady job (like in te first part) and the distribution strategy of deliveries.

Thus, the research goals are:
\begin{itemize}
 \item \textbf{Building the above-mentioned model in NetLogo, creating it as a highly abstracted but still believable world.}
 \item \textbf{Finding how the profit for a delivery company varies when using independent contractors versus hiring deliverers as employees.}
\end{itemize}

Answer will be of the kind:

time series: profits over time, averages of multiple runs


\todo[inline]{Write more about this \& author.}
