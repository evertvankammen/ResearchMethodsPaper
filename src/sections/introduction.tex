\section{Introduction}\label{sec:introduction}
Food delivery companies, like Uber Eats and Just Eat Takeaway, offer a service where customers can order food from restaurants and have the meal delivered to their homes.
These companies get a fee per delivery and pay deliverers an hourly wage or pay them per delivery.

E.g., Uber Eats~\footnote{\url{merchants.ubereats.com/gb/en/pricing/}} charges 30\% of the total value of an ordered meal to be paid by the restaurant.
Deliverers get paid per delivery (this is not true in all countries), which consists of base pay and an optional tip from the customer.
Drivers may decide whether to take an order; the delivery companies see them as independent contractors.

Lately, delivery companies have had bad publicity for being unfair and exploiting the deliverers.
In the Netherlands, the government calls the status of these deliverers bogus self-employment.
That is why, in the Netherlands, delivery companies are forced to hire them as employees with all benefits and pay an hourly wage.
For this reason, Deliveroo, a British company, even left the Netherlands market.
However, other countries still see them as independent workers and pay them per delivery.

Delivery companies prefer the per-delivery system as they do not have to pay for the time they make no deliveries.
This system may also have advantages for the deliverers; some may make more money if they can make many deliveries.
However, if deliverers don't make enough money, they may quit, and thus, fewer deliveries can be made, leading to lesser profits for the delivery companies.

In this research project, different scenarios are simulated, such as deliverers delivering ordered meals to customers.
The scenarios differ in how deliverers are treated: independent contractors who get paid per delivery versus
those hired as employees.
Some other variables are also examined: how meals are distributed among deliverers, the mean delivery preparation times, and how deliveries are distributed among deliverers.
In these scenarios, the company's total number of deliveries will be calculated, and the number of participating restaurants, deliverers, and customers will be analyzed.

This is an economic problem that has both micro and macroeconomic facets.
Microeconomic as it deals with agents making decisions, e.g., hiring deliverers, maximizing the delivery companies' deliveries, ordering food, becoming a deliverer, or deciding to stop being a deliverer.

But macroeconomic elements are also present, but not in the conventional sense.
Macroeconomics studies aggregate and emerging statistical regularities in the economy~\cite{cincotti2022we}.
The Agent-Based Model shows the results of the behavior of many individuals as a whole, resulting from the behavior and interaction of heterogeneous agents.
This differs from conventional macroeconomics, which sees agents as having the behavior of a `typical' that is `representative' or `average' individual.

A model will, like in a laboratory setting, include only the properties we are interested in and exclude elements like weather and the physical conditions of deliverers.

The tool for building the model is NetLogo ~\cite{NetLogo2024}.
This is a multi-agent programmable modeling environment.
This environment has been used to create models for many sciences concerning multi-agents (from how viruses spread to how agents act in game theory~\cite{r2019agent}).

The environment where multiple agents act is called a multi-agent system, and the problem is called a multi-agent planning problem (from~\cite{russell2016artificial}).

The stakeholders are researchers (in e.g., economics, multi-agent systems research) interested in using ABM models in their research specially for the study of the last mile in the delivery industry.
These researchers may adapt the model to include other rules, agents etc.
Also, it could give decision makers in the delivery industry insight in the differences between the two employment methods.

\subsection{Research goals}\label{subsec:research-goals}
This research is not thoroughly theoretical; it's more of an exploratory/explanatory nature of ABM in the setting of the last mile of the food delivery industry.

The main research goal is to create an Agent-Based Model in NetLogo to compare two deliverer employment scenarios in the food delivery industry.

The two main scenarios of interest are:
\begin{enumerate}
\item A scenario where the deliverers are hired as employees.
\item A scenario where deliverers are independent contractors and get paid per delivery.
\end{enumerate}

Thus, the research goals are:

\textbf{Building an Agent Bases Model in NetLogo for the food delivery process, as a highly abstracted but still believable world.}

\textbf{Finding how the number of deliveries for a delivery company varies when using independent contractors versus hiring deliverers as employees.
 And what is the relative importance of the model variables?}

The model will be built incrementally; more about this in the methods section.

The second step is executed from the situation where deliverers are hired, and thus, their numbers are stable.
Then, find the effects of several variables and find settings where the model is stable.

From the stable settings, determine the average deliveries per deliverer per time unit.

Given these stable settings, run the model a second time, but now the deliverers may leave when their deliveries are less
compared to the first run.

The differences are the answers to the second research goal.


