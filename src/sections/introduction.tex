
\section{Introduction}

Food delivery companies, like Uber Eats and Thuisbezorgd.nl, offer a service where customers can order food from restaurants and get the meal delivered at home.
These companies get a fee per delivery and pay deliverers an hourly wage and/or pay them per delivery.

E.g., Uber Eats ~\footnote{\url{https://merchants.ubereats.com/gb/en/pricing/}} charges 30\% of the total value of an ordered meal to be paid by the restaurant.
Deliverers get paid per delivery (this is not true in all countries) this consists of a base pay and an optional tip from the customer.
Drivers may decide to take an order or not, they are seen as independent contractors.

Lately delivery companies have had some bad publicity of being unfair and exploiting the deliverers.
In the Netherlands the government calls the status of these deliverers: bogus self-employment.
That is why in the Netherlands delivery companies are forced to hire them as employees with all benefits and pay an hourly wage.
But in other countries they still see them as independent workers and pay them per delivery.

Delivery companies seem to prefer the per-delivery system as they do not have to pay for the time no deliveries are made.
This system may have advantages for the deliverers too, some may make more money if they can make many deliveries.
But, if deliverers don't make enough money they may quit and thus fewer deliveries can be made leading to lesser profits for the delivery companies.

In this research project different scenarios are simulated where deliverers deliver ordered meals to customers.
The scenarios differ in how deliverers are treated: as independent contractors who get paid per delivery and thus may decide to take an order or not versus
deliverers hired as employees where orders are distributed among available deliverers.
In the context of these scenarios the profit of the company will be calculated, also the number of participating restaurants, deliverers and customers will analyse.

As the research budget is not large enough to try this in the real world a model will be build that will run some simulations.
A model will also, like in a laboratory setting, include only properties we are interested in and exclude elements like e.g., weather and physical conditions of deliverers.
The tool for building the model is: NetLogo ~\cite{NetLogo2024}.
This is a multi-agent programmable modeling environment.
This environment has been used to create models for many sciences concerning multi-agents (from how viruses spread to how agents act in game theory~\cite{r2019agent}).

Other research projects reported about the food delivery sector using the NetLogo tool.
An existing model for the food delivery context is created by~\cite{ismail2024software}.
This model was used to research the efficiency of food delivery.
The model consists riders, vendors and customers.
The world consists of a 2-dimensional grid where riders have to pick up food from vendors and deliver it to customers.
This project also mentioned some real world rules the 3 types of agents possess.

Another use of NetLogo is ~\cite{chella2023quantum} where agents, i.e., in this case a swarm of robots, collectively have to find a shortest path.
This could also be used by the food delivery environment.

In another study ~\cite{antelmi2024reliable} NetLogo is compared to other ABM systems and integrations with other programming languages are analysed.



\section{Research goals}

The main research goal for this research is to create an Agent Based model to compare several deliverer employment scenarios  in the food delivery industry.
The two main scenarios of interest are:
    - where the deliverers are hired as employees
    - versus a scenario where deliverers are independent contractors and get paid per delivery

This model is of the Agent Based type.


\todo[inline]{Get info about ABM \& author.}

\cite{hamill2016agent}




\textbf{How does the profit for delivery companies vary when using independent contractors versus hiring deliverers as employees with an hourly wage? }

An instrument will be built in the form of an Agent Based Model.
Calling it an instrument does not mean it is not an important deliverable, as most time and energy will be spent in creating it.

Thus, the second goal of the research is: building an Agent Based Model (ABM) that reflect a world where 3 types of agents exist: restaurants, customers and deliverers.

